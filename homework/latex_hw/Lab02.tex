\documentclass[11pt]{article}
\usepackage{amsthm,amssymb,graphicx,graphicx,multirow,amsmath,cite}
\usepackage[usenames,dvipsnames]{color}
\usepackage{epstopdf}
\usepackage{etoolbox}
\include{notation}
\oddsidemargin=0.15in
\evensidemargin=0.15in
\topmargin=-.5in
\textheight=9in
\textwidth=6.25in
%\bootrue{Advanced}

\newtheorem{lemma}{Lemma}{}
  \newtheorem{thm}{Theorem}
  \newtheorem{theorem}{Theorem}
  \newtheorem{prop}{Proposition}
  \newtheorem{cor}[thm]{Corollary}
  \newtheorem{defi}[thm]{Definition}
  \newtheorem{definition}[thm]{Definition}
  
  %% USE THESE IN YOUR TEX CODE.
\def\E{\mathbb{E}} % For Expectation
\def\P{\mathbb{P}} % for probabiltiy
\def\EE{\mathbb{E}^{!o}} % For Palm expectation
\def\ie{{\em i.e.}} 
\def\eg{{\em e.g.}}
\def\V{\operatorname{Var}}
\def\L{\mathcal{L}} % For Laplace transform
\def\i{\mathbf{1}} % Indicator random variable
\def\l{\ell}% For path loss moel

\begin{document}
%--------------
%% preamble.tex
%% this should be included with a command like
%% %--------------
%% preamble.tex
%% this should be included with a command like
%% %--------------
%% preamble.tex
%% this should be included with a command like
%% \input{preamble.tex}
%% \lecture{1}{September 4, 1996 }{Daniel A. Spielman}{name
%%  of poor scribe}

\hbadness=10000
\vbadness=10000

%\setlength{\oddsidemargin}{.25in}
%\setlength{\evensidemargin}{.25in}
%\setlength{\textwidth}{6in}
%\setlength{\topmargin}{-0.4in}
%\setlength{\textheight}{8.5in}

\newcommand{\handout}[5]{
   \renewcommand{\thepage}{#1-\arabic{page}}
   \noindent
   \begin{center}
   \framebox{
      \vbox{
    \hbox to 5.78in { {\b ECEN5322: Search Engines and Analysis of High Dimensional Datasets

 }
     	 \hfill #2 }
       \vspace{4mm}
       \hbox to 5.78in { {\Large \hfill #5  \hfill} }
       \vspace{2mm}
       \hbox to 5.78in { {\it #3 \hfill #4} }
      }
   }
   \end{center}
   \vspace*{4mm}
}

\newcommand{\lecture}[4]{\handout{#1}{#2}{ #3}{Scribe: #4}{ Lecture #1}}
\newcommand{\assignment}[5]{\handout{#1}{#2}{ #3}{Author: #4(#5)}{ Assignment #1}}
\newcommand{\labreport}[5]{\handout{#1}{#2}{ #3}{Author: #4(#5)}{ Lab Report #1}}
%% \lecture{1}{September 4, 1996 }{Daniel A. Spielman}{name
%%  of poor scribe}

\hbadness=10000
\vbadness=10000

%\setlength{\oddsidemargin}{.25in}
%\setlength{\evensidemargin}{.25in}
%\setlength{\textwidth}{6in}
%\setlength{\topmargin}{-0.4in}
%\setlength{\textheight}{8.5in}

\newcommand{\handout}[5]{
   \renewcommand{\thepage}{#1-\arabic{page}}
   \noindent
   \begin{center}
   \framebox{
      \vbox{
    \hbox to 5.78in { {\b ECEN5322: Search Engines and Analysis of High Dimensional Datasets

 }
     	 \hfill #2 }
       \vspace{4mm}
       \hbox to 5.78in { {\Large \hfill #5  \hfill} }
       \vspace{2mm}
       \hbox to 5.78in { {\it #3 \hfill #4} }
      }
   }
   \end{center}
   \vspace*{4mm}
}

\newcommand{\lecture}[4]{\handout{#1}{#2}{ #3}{Scribe: #4}{ Lecture #1}}
\newcommand{\assignment}[5]{\handout{#1}{#2}{ #3}{Author: #4(#5)}{ Assignment #1}}
\newcommand{\labreport}[5]{\handout{#1}{#2}{ #3}{Author: #4(#5)}{ Lab Report #1}}
%% \lecture{1}{September 4, 1996 }{Daniel A. Spielman}{name
%%  of poor scribe}

\hbadness=10000
\vbadness=10000

%\setlength{\oddsidemargin}{.25in}
%\setlength{\evensidemargin}{.25in}
%\setlength{\textwidth}{6in}
%\setlength{\topmargin}{-0.4in}
%\setlength{\textheight}{8.5in}

\newcommand{\handout}[5]{
   \renewcommand{\thepage}{#1-\arabic{page}}
   \noindent
   \begin{center}
   \framebox{
      \vbox{
    \hbox to 5.78in { {\b ECEN5322: Search Engines and Analysis of High Dimensional Datasets

 }
     	 \hfill #2 }
       \vspace{4mm}
       \hbox to 5.78in { {\Large \hfill #5  \hfill} }
       \vspace{2mm}
       \hbox to 5.78in { {\it #3 \hfill #4} }
      }
   }
   \end{center}
   \vspace*{4mm}
}

\newcommand{\lecture}[4]{\handout{#1}{#2}{ #3}{Scribe: #4}{ Lecture #1}}
\newcommand{\assignment}[5]{\handout{#1}{#2}{ #3}{Author: #4(#5)}{ Assignment #1}}
\newcommand{\labreport}[5]{\handout{#1}{#2}{ #3}{Author: #4(#5)}{ Lab Report #1}}
\labreport{2}{09/27/2015}{The Group Structure of $S^{n-1}$}{Prasanth Prahladan}{100817764}


\textbf{1. Prove that O(n) is a group when equipped with matrix multiplication.}\\
Let $U_1,U_2 \in O(n)$. Therefore, by definition we have $U_i^T*U_i = U_i*U_i^T = I_n$. Define $P = U_1*U_2$ and $Q = U_2*U_1$. 
\begin{align*}
P^T*P = (U_1*U_2)^T*(U_1*U_2) = (U_2^T*U_1^T)*(U_1*U_2) = U_2^T*(U_1^T*(U_1)*U_2 = I_n
\end{align*}
Similarly, we can prove $P*P^T = I_n$ and $Q*Q^T = Q^T*Q = I_n$. Therefore, $P, Q \in O(n)$. Hence, O(n) is a group when equipped with the matrix multiplication operation.

\hrulefill

\textbf{2. Prove that $U \in O(n) \implies det(U) = \pm 1$.}\\
Since, $U_i^T*U_i = U_i*U_i^T = I_n$ and $det(MN) = det(M)det(N)$ we have 
\begin{align*}
det(U)^2 = det(I_n) = 1 \implies det(U) = \pm 1.
\end{align*}

\hrulefill

\textbf{3. Prove that SO(n) is a subgroup of O(n).}\\
Let $U_1, U_2 \in SO(n)$. Therefore, $det(U_i) = 1$ and $U_i^T*U_i = U_i*U_i^T = I_n$.Define $P = U_1*U_2$ and $Q = U_2*U_1$. \begin{align*}
det(P) = det(Q) &= det(U_1)*det(U_2) = 1\\
P^T*P = P*P^T &= I_n = Q^T*Q = Q*Q^T 
\end{align*}
Therefore, $P,Q \in SO(n)$.
Further, we know that $U_i^{-1} \in SO(n)$ such that $U_i^{-1}*U_i = I_n \in SO(n)$. Therefore, $\tilde{Q} =  U_1^{-1}*U_2^{-1} \in SO(n)$ and $\tilde{P} = U_2^{-1}*U_1^{-1} \in SO(n)$.
\begin{align*}
P*\tilde{P} =  (U_1*U_2)*(U_2^{-1}*U_1^{-1}) & = U_1*(I_n)*U_1^{-1} = I_n\\
Q*\tilde{Q} &= I_n
\end{align*} 
which proves that, $\tilde{P} = P^{-1}$ and $\tilde{Q} = Q^{-1}$. Thus, SO(n) is a subgroup of O(n).

\hrulefill

\textbf{4. Consider the subset $\mathcal{G}$ of SO(n) defined by
\begin{align*}
\mathcal{G} = {U \in SO(n) , Ue_1 = e_1}
\end{align*}
where $e_1$ is the first element of the canonical basis in $\mathcal{R}^n$. Prove that $\mathcal{G}$ is a subgroup of SO(n).}\\
Let $G_1, G_2 \in \mathcal{G} \subset SO(n)$. Therefore, $G_i e_1 = e_1$. Define $P = G_1*G_2 \in SO(n)$.
\begin{align*}
Pe_1 &= (G_1*G_2)e_1 = G_1(e_1) = e_1\\
\end{align*} Therefore, $P \in \mathcal{G}$.
Further, we observe that
\begin{align*}
e_1 &= G_ie_1\\
G_i^{-1}e_1 &= G_i^{-1}*G_i e_1 = I_ne_1 = e_1
\end{align*}
Therefore, $G_i^{-1} \in \mathcal{G}$ and
\begin{align*}
P^{-1}e_1 = (G_1*G_2)^{-1}e_1 = G_2^{-1}*G_1^{-1}e_1 = G_2^{-1}e_1 = e_1
\end{align*}
This implies that $P \in \mathcal{G}$ and hence, $\mathcal{G}$ is a Subgroup of SO(n).

\hrulefill

\textbf{5. Prove that every element $U \in \mathcal{G}$ can be written as below with $W \in SO(n-1)$
\begin{align}
U = \bigg[
\begin{array}{cc}
1 & 0\\
0 & W
\end{array}
\bigg] \label{eq:modifiedU}
\end{align}
}
Let $\{ u_1 , u_2, u_3, \cdots u_n \}$ be the n-column vectors defining $U =[ u_1 u_2 \cdots u_n]\in \mathcal{G}$. We have, $Ue_1 = e_1 \implies u_1 = e_1$. Since, rank$(U) = n$, and $\{u_i\}$ are all orthogonal vectors, we know that $ \{u_2, \cdots u_n\}$ form a set of (n-1) independent vectors. Its possible to determine a rotation($K \in SO(n)$) of these (n-1) vectors such that their first-component is 0. Therefore, $K[u_i] = [0 v_i]'$ where $\{v_i\}$ form a set of (n-1) linearly independent orthogonal (n-1) dimensional vectors.
\begin{align*}
U = \bigg[\begin{array}{cc}
1 & \bar{x}'\\
\bar{0} & M
\end{array}\bigg] &\xrightarrow{K \in SO(n)}
\bigg[\begin{array}{cc}
1 & \bar{0}'\\
\bar{0} & W
\end{array}\bigg] = V \in \mathcal{G}
\end{align*} where $\bar{x} \in \mathcal{R}^{n-1}$. Further, note that
\begin{align*}
det(V) &= det(KU) = det(K)*det(U) = 1\\
&= 1.det(W)\\
V^T*V  &= \bigg[\begin{array}{cc}
1 & \bar{0}'\\
\bar{0} & W^TW  
\end{array} \bigg] = I_n
\end{align*}
Therefore, $W \in (n-1)\times(n-1)$ real matrix with $det(W) = 1$ and $ W^T * W = W * W^T = I_{n-1}$. Hence, $W \in SO(n-1)$.

\hrulefill

\textbf{6. Prove that $\Psi: \mathcal{G} \rightarrow SO(n-1)$ such that $\Psi(U) = W$ in \eqref{eq:modifiedU} is an 'isomorphism' between $\mathcal{G}$ and SO(n-1).}\\
Consider elements $U_1, U_2 \in \mathcal{G}$ and $\Psi(U_i) = W_i$ where $W_i \in SO(n-1)$, $U_i \in \mathcal{G}$ and
\begin{align}
U_i &= \bigg[\begin{array}{cc}
1 & \bar{0}'\\
\bar{0} & W_i
\end{array}\bigg]. 
\end{align}.
We can make the following observations
\begin{align*}
U_1*U_2 &= \bigg[\begin{array}{cc}
1 & \bar{0}'\\
\bar{0} & W_1
\end{array}\bigg]*
\bigg[\begin{array}{cc}
1 & \bar{0}'\\
\bar{0} & W_2
\end{array}\bigg]\\  
&= \bigg[\begin{array}{cc}
1 & \bar{0}'\\
\bar{0} & W_1*W_2
\end{array}\bigg]\\
\Psi(U_1*U_2) &= W_1*W_2 = \Psi(U_1)*\Psi(U_2)
\end{align*}

\begin{enumerate}
\item \emph{Surjective:} $\forall W \in SO(n-1)$ we can construct $U$ such that
\begin{align}
U &= \bigg[\begin{array}{cc}
1 & \bar{0}'\\
\bar{0} & W
\end{array}\bigg]. 
\end{align}
Note that, $det(U) = 1.det(W) = 1$, $U^TU = UU^T = I_n$ and $Ue_1 = e_1$. Therefore, $U \in \mathcal{G}$ and 
$\Psi(\cdot)$ is a surjective function.
\item \emph{Injective:} Let $W_1 = W_2$ where $W_i = \Psi(U_i)$, $W_i \in SO(n-1)$ and $U_i \in \mathcal{G}$. Therefore,
\begin{align*}
\Psi(W_1) &= \Psi(W_2)\\
\bigg[\begin{array}{cc}
1 & \bar{0}'\\
\bar{0} & W_1
\end{array}\bigg] &= \bigg[\begin{array}{cc}
1 & \bar{0}'\\
\bar{0} & W_2
\end{array}\bigg]\\
U_1 &= U_2\\
\Psi(U_1) = \Psi(U_2) &\implies U_1 = U_2
\end{align*}
Hence, $\Psi(\cdot)$ is an injective function.
\end{enumerate}
From the discussion above, we prove that $\Psi(\cdot)$ is a bijective and an Isomorphism between $\mathcal{G}$ and SO(n-1).

\hrulefill

\textbf{7. Prove that $\forall x \in S^{n-1} \exists U \in SO(n)$ such that $Ue_1 = x$. Is U unique? }\\
Let $\{u_1, u_2, \cdots u_n\}$ be the columns of the matrix $U \in SO(n)$. If $u_1 = x$, then we obtain $Ue_1 = x$. Therefore, we also require that the n-1 column vectors $\{u_2,\cdots u_n\}$ be orthogonal to x.
\begin{align*}
U = \bigg[\begin{array}{cccc}
x &u_2 &\cdots &u_n
\end{array}\bigg] \in SO(n)
\end{align*}
Note, that any rotation of the matrices $u_{i} \neq x$ can be used to construct U. Hence, $U \in SO(n)$ is not unique.



\hrulefill

\textbf{8. Prove that if $U,V \in SO(n)$ such that $Ue_1 = Ve_1 = x$, then there exits $W \in SO(n-1)$ such that
\begin{align*}
U = V*\bigg[\begin{array}{cc}
1 & \bar{0}'\\
\bar{0} & W
\end{array}\bigg]
\end{align*}
}
\\
Let $G \in \mathcal{G} \subset SO(n)$. From above, we know that $\exists W \in SO(n-1)$
\begin{align*}
Ge_1 &= e_1\\
G &= \bigg[\begin{array}{cc}
1 & \bar{0}'\\
\bar{0} & W
\end{array}\bigg]
\end{align*}
We make the following observations
\begin{align*}
x &= Ve_1 = V*(G e_1)\\
&= U e_1\\
U &= V*G = V*\bigg[\begin{array}{cc}
1 & \bar{0}'\\
\bar{0} & W
\end{array}\bigg]
\end{align*}

\hrulefill

\textbf{9. Prove that if $U_1$ and $U_2$ are two coset representatives for the same coset then $U_1 \mathcal{G} = U_2 \mathcal{G}$.}\\
$U\mathcal{G}$ is the coset defined by $U \in SO(n)$ and contains $V = UG, \forall G\in \mathcal{G}$.
The set of all cosets $\{U\mathcal{G}\} = SO(n)/\mathcal{G}$, is called the 'quotient-group'.

Let $U\mathcal{G}$ be the coset under consideration, with coset-representations $V_1, V_2 \in U\mathcal{G}$. Therefore, by definition we have for $G_1, G_2 \in \mathcal{G}$
\begin{align*}
U = V_1 G_1 = V_2 G_2\\
\end{align*}
Further, 
\begin{align*}
V_1\mathcal{G} = \{ X | V_1 = X G, \forall X \in SO(n), \forall G \in \mathcal{G} \text{ and } V_1 \in SO(n)\}\\
V_2\mathcal{G} = \{ Y | V_2 = Y G, \forall Y \in SO(n), \forall G \in \mathcal{G} \text{ and } V_2 \in SO(n)\}
\end{align*}
 We note the following
 \begin{align*}
 V_1\mathcal{G} &= \{ X | V_1 = X G, \forall X \in SO(n)\}\\ 
 &= \{ X | V_1G_1 = XGG_1, \forall X \in SO(n)\}\\ 
    &= \{ X | V_2G_2 = XGG_1, \forall X \in SO(n)\}\\ 
        &= \{ X | V_2 = XGG_1G_2^{-1}, \forall X \in SO(n)\}\\
                &= \{ X | V_2 = X\tilde{G}, \forall X \in SO(n)\}\\
                &= V_2\mathcal{G}
 \end{align*}


\hrulefill

\textbf{10. Define the map $\Phi:SO(n)/\mathcal{G} \rightarrow S^{n-1}$ such that $U \mapsto \Phi(U) = Ue_1$. Prove that $\Phi(\cdot)$ represents the action of SO(n) i.e. $\forall \Omega \in SO(n), \forall U \in SO(n)/\mathcal{G}$ 
\begin{align*}
\Phi(\Omega U )= \Omega \Phi(U)
\end{align*}
}
Let $x \in S^{n-1}$. We know $\exists U \in SO(n)/\mathcal{G}$ such that $Ue_1 = x$. By definition, $\Phi(U) = Ue_1 = x$
\begin{align*}
\Phi(\Omega U) = (\Omega U) e_1 &= \Omega(U e_1)= \Omega \Phi(U)
\end{align*}
Hence, $\Phi(\cdot)$ represents the action of SO(n).


\hrulefill

\textbf{11. Prove that the map $\Phi(\cdot)$ is bijective and continuous.}\\
To prove that $\Phi(\cdot)$ is bijective, we proceed as follows
\begin{enumerate}
\item \emph{Injective} Let $x_1 = x_2 = \tilde{x}$, where $\Phi(U_i) = U_ie_1 = x_i$. Therefore, we have
\begin{align*}
x_1 &= x_2\\
\Phi(U_1) &= \Phi(U_2)\\
U_1 e_1 &= U_2 e_1 \\
&= U_2 G e_1
\end{align*} where $G \in \mathcal{G}$. Therefore, $U_1,U_2 \in U_1\mathcal{G}$ i.e they belong to the same equivalence class. Therefore,there exits a unique $\tilde{U} \in SO(n)/\mathcal{G}$ such that $U_1 = U_2 = \tilde{U}$.
Hence, $\Phi(\cdot)$ is injective.
\item \emph{Surjective}
For any $x \in S^{n-1}$, there exists non-unique $U \in SO(n)$ such that $Ue_1 = x$. By using the equivalence rule, $U = VG, G\in \mathcal{G}$, there exists a unique $U \in SO(n)/\mathcal{G} \forall x$ such that $Ue_1 = x$. Therefore, $\Phi(\cdot)$ is a surjective function. 
\end{enumerate}
From the above discussion, we learn that $\Phi(\cdot)$ is a bijective function. 

To prove that $\Phi(\cdot)$ is continuous, we use the notion that images of a converging sequence of points in the domain, form a converging sequence in the range of the map i.e. 
\begin{align*}
\lim_{n\to \infty} \Omega_n \rightarrow \Omega \implies \lim_{n\to \infty} \Phi(\Omega_n)\text{ exists}.
\end{align*}

We define convergence in the domain, using the Frobenius norm i.e.
\begin{align*}
\lim_{n\to \infty} \Omega_n \rightarrow \Omega \implies \lim_{n\to \infty} ||\Omega_n - \Omega||_{F} \rightarrow 0
\end{align*}

We observe that
\begin{align*}
\lim_{n\to \infty} || \Phi(\Omega_{n+1}) - \Phi(\Omega_{n})|| &= \lim_{n\to \infty} || \Omega_{n+1}e_1 - \Omega_n e_1||\\
&\leq \lim_{n\to \infty} ||\Omega_{n+1} - \Omega_n|| ||e_1|| \rightarrow 0
\end{align*}
Therefore, $\lim_{n\to \infty} \Phi(\Omega_{n})$ exists! 
Hence, $\Phi(\cdot)$ is a continuous mapping.


\hrulefill

Conclusion: 
\begin{align*}
\mathcal{G} &\xrightarrow{isomorphism} SO(n-1)\\
SO(n)/\mathcal{G} &\leftrightarrow SO(n)/SO(n-1)\\
S^{n-1} &\xrightarrow{homeomorphism} SO(n)/\mathcal{G}\\ 
S^{n-1} &\leftrightarrow SO(n)/SO(n-1)\\
\end{align*}

We thus conclude that the Sphere $S^{n-1}$ can be identified with the Quotient Group, $SO(n)/SO(n-1)$. And thus properties of Group Action, Group Symmetries can be used to understand it. Further, the topological Group structure of the Sphere, permits us to define a Haar Measure on it, which shall then be used to measure the set of points under study.

\hrulefill

\hrulefill

\definition[Group]{A group is non-empty set G with an operation * and satisfying the following properties
\begin{enumerate}
\item for any $f, g\in G$, $h*g \in G$
\item for any $f, g, h \in G$, $f*(g*h) = (f*g)*h$
\item $\exists e \in G$ such that for any $g \in G$, $g*e = e*g = g$
\item for any $g \in G$, $\exists g^{-1} \in G$, such that $g*g^{-1} = g^{-1}*g = e$
\end{enumerate}

\definition[Subgroup]{A subset H of G is a subgroup if 
\begin{enumerate}
\item for any $g,h \in H$, $h*g \in H$
\item for any $g \in H$, $\exists g^{-1} \in H$, such that $g*g^{-1} = g^{-1}*g = e$
\end{enumerate}

\definition[Orthogonal Group, O(n)]{
The orthogonal group $O(n)$ is the subset of $n\times n$ real matrices U such that $U^T*U = U*U^T = I_n$, where $I_n$ is the $n\times n$ Identity Matrix.
}

\definition[Special Orthogonal Group, SO(n)]{
SO(n) consists of the elements of O(n) with determinant 1 i.e. the rotations.
}

\definition[Homomorphism]{
A function $\phi: (G,*) \rightarrow (F,\Delta)$ is called a homomorphism if $\phi$ commutes with Group Operations
\begin{align*}
\forall G_1, G_2 \in G, \phi(G_1*G_2) = \phi(G_1)\Delta \phi(G_2)	
\end{align*} 
}

\definition[Isomorphism]{ Let $f:(G,*)\rightarrow (F,\Delta)$ be a homomorphism. f is an ismomorphism if f is bijective i.e $f^{-1}$ exits and $f^{-1}$ is also a homomorphism.
}

\definition[Equivalence Class]{ Two matrices U and V are equivalent under a relation if $Ue_1 = Ve_1 = x$, or equivalently there exists $Q \in \mathcal{G}$ such that $U = VQ$. Let $U \in SO(n)$. The Left-Coset is defined as $U\mathcal{G} = \{V| V\in SO(n-1), V = UQ, Q \in \mathcal{G}\}$} where V is called 'coset representative'.

\definition[Quotient Group]{The set of left-cosets is called Quotient Group, denoted by $SO(n)/\mathcal{G}$.}
 
\end{document}




