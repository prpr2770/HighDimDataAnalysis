\documentclass[9pt]{article}
\usepackage{amsthm,amssymb,graphicx,graphicx,multirow,amsmath,cite}
\usepackage[usenames,dvipsnames]{color}
\usepackage{epstopdf}
\usepackage{etoolbox}

% coloring code snippets
\usepackage{listings}
\usepackage{color} %red, green, blue, yellow, cyan, magenta, black, white
\definecolor{mygreen}{RGB}{28,172,0} % color values Red, Green, Blue
\definecolor{mylilas}{RGB}{170,55,241}

\include{notation}
\oddsidemargin=0.15in
\evensidemargin=0.15in
\topmargin=-.5in
\textheight=9in
\textwidth=6.25in
%\bootrue{Advanced}

\newtheorem{lemma}{Lemma}{}
  \newtheorem{thm}{Theorem}
  \newtheorem{theorem}{Theorem}
  \newtheorem{prop}{Proposition}
  \newtheorem{cor}[thm]{Corollary}
  \newtheorem{defi}[thm]{Definition}
  \newtheorem{definition}[thm]{Definition}
  
  %% USE THESE IN YOUR TEX CODE.
\def\E{\mathbb{E}} % For Expectation
\def\P{\mathbb{P}} % for probabiltiy
\def\EE{\mathbb{E}^{!o}} % For Palm expectation
\def\ie{{\em i.e.}} 
\def\eg{{\em e.g.}}
\def\V{\operatorname{Var}}
\def\L{\mathcal{L}} % For Laplace transform
\def\i{\mathbf{1}} % Indicator random variable
\def\l{\ell}% For path loss moel

\begin{document}
\lstset{language=Matlab,%
	basicstyle=\footnotesize,  
    breaklines=true,%
    morekeywords={matlab2tikz},
    keywordstyle=\color{blue},%
    morekeywords=[2]{1}, keywordstyle=[2]{\color{black}},
    identifierstyle=\color{black},%
    stringstyle=\color{mylilas},
    commentstyle=\color{mygreen},%
    showstringspaces=false,%without this there will be a symbol in the places where there is a space
    numbers=left,%
    numberstyle={\tiny \color{black}},% size of the numbers
    numbersep=9pt, % this defines how far the numbers are from the text
    emph=[1]{for,end,break},emphstyle=[1]\color{red}, %some words to emphasise
    %emph=[2]{word1,word2}, emphstyle=[2]{style},    
}
 
\input{preamble.tex}
\labreport{3}{10/30/2015}{Community Detection in Networks}{Prasanth Prahladan}{100817764}

\section{Experiments}

\subsection{Planted Partition Model}

\textbf{Q 13. Matlab function that takes p,q,n as input and generates the adjacency matrix of the planted partition model. } 

\begin{lstlisting}
function [A, partitionIndicatorVec] = getPartitionGraphModel(n,p,q)
%{ 
Q 13.
n : total #nodes in G
p : probability of link between two vertices inside Cluster
q : probability of link edges between two vertices in opposite clusters
%}

if (mod(n,2) == 0 ) % n is EVEN
%     generate Permutation Matrix, T
    I = eye(n);
    ix = randperm (n);
    T = I(ix,:);

%     generate adjaceny matrix A of a planted partition over n nodes
    n2 = n/2;
    P = random('bino', 1, p, n2, n2); % upper left block
    dP2 = random('bino', 1, p, n2, 1); % diagonal of the lower right block
    Q = random('bino', 1, q, n2, n2); % upper right block
%     carve the two triangular and diagonal matrices that we need
    U = triu(P, 1);
    L = tril(P,-1);
    dP = diag(P);
    B0 = U + U' + diag(dP);
    B1 = Q;
    B2 = Q';
    B3 = L + L' + diag(dP2);
    B =[B0 B1;B2 B3];

    comm1 = ones(1,n2);
    originalCluster = [comm1 -1*comm1];
    

%  PERMUTE THE NODES
%     Re-index Nodes of the graph. 
%     B*T' -> exchg columns; T*M -> exchg rows
    A = T*B*T'; 

%   Obtain the True_Cluster_NodeID for Graph A: Permute them
    partitionIndicatorVec  = originalCluster(ix);
% % -----------------------------------------
else
   warning('n == ODD!') ;
end
end

\end{lstlisting}

\hrulefill



\textbf{Q 14.  Implementation of Partition Algorithm}
\begin{lstlisting}
function partitionIndicatorVec = runPartitionAlgo(A)
%{ 
Algorithm Partition
* compute the second dominant eigenvector, v2, of A, associated with the second largest
eigenvalue ?2.
* for i = 1 to n
if the coordinate i of v2 is positive,(v2)_i > 0, then  %what if its' ZERO?
assign node wi to community 1
else
assign node i to community 2.
end
end

partitionIndicatorVec : = {1(partition A), -1(partition B)}

%}


[V, D] = eigs(A,2);

vec = V(:,2)';
pos = (vec >0);
neg = (vec <0);
partitionIndicatorVec = pos - neg;

end
\end{lstlisting}
\hrulefill


\textbf{Q 15. Computing Overlap between the True-Partition and Predicted/Estimated-Partitions} 

\begin{align}
\omega_i &= \bigg\lbrace \begin{array}{cc}
1 & \text{if i belongs to partition 1}\\
-1 & \text{if i belongs to partition 2}
\end{array} \label{eq:truePartition}\\
\tilde{\omega_i}&= \bigg\lbrace \begin{array}{cc}
1 & \text{if } (v_2)_i > 0,\\
-1 & \text{otherwise}
\end{array} \label{eq:estPartition}\\ 
rawoverlap &= max \bigg( \sum_{i=1}^n  \delta_{\omega_i, \tilde{\omega_i}} ,  \sum_{i=1}^n  \delta_{-\omega_i, \tilde{\omega_i}}   \bigg) \label{eq:rawoverlap}\\
overlap &= \frac{2}{n}rawoverlap - 1 \label{eq:overlap}
\end{align}

(a) Compute overlap score when $\tilde{\omega} = \omega$.\\

From \eqref{eq:rawoverlap} we obtain $rawoverlap = n$ when $\tilde{\omega} = \omega$. Substituting into \eqref{eq:overlap}, we obtain
\begin{align*}
overlap = \frac{2}{n}(n) - 1 = 1
\end{align*}

(b) Prove that a random guess for the detection of the communities returns overlap 0.\\
A random guess for the detection of communities is a Binary vector of $\{-1,1\}^n$ with each component chosen with equal probability $(0.5)$. From the definition \eqref{eq:rawoverlap}, we obtain $rawoverlap = \frac{n}{2}$. Thus, we obtain
\begin{align*}
overlap = \frac{2}{n}\frac{n}{2}  - 1 = 0.
\end{align*}

\begin{lstlisting}
function  overlap = getPartitionOverlap(w1, w2)
%{
Q15. 
Derive the Overlap Metric based on the 
w1 : true partition vector
w2 : estimated partition vector
%}
n = length(w1);
del_w1_w2 = sum(w1 == w2);
del_minus_w1_w2 = sum(-w1 == w2);

rawoverlap = max(del_w1_w2, del_minus_w1_w2);

% random choice of w2 generates a non-zero overlap. Accounting for this:
overlap = (2/n)*rawoverlap - 1;         % Interpret: Prob. of successfully detecting communities.
end

\end{lstlisting}

\hrulefill


\textbf{Q 16/17. Dense and Sparse Communities}



\begin{figure}[h!]
\centering
\includegraphics[scale=0.2]{hw3q16_denseMatrix_final.jpg}\\
\caption{Dense Network: Probability of successfully detecting the partitions using the Partition-Algorithm. The x-axis corresponds to $\beta$ and the y-axis corresponds to $\alpha$. The decision boundaries are overlaid in red. The community-recovery algorithm is implemented only for case where $0\leq q<p \leq 1<$. The decision boundary lying inside the Dark region has $(\alpha,\beta)$ values that violate the $q<p$ requirement, and hence, can be ignored. }
\label{fig:DenseNetwork}
\end{figure}




\begin{figure}[h!]
\centering
\includegraphics[scale=0.2]{hw3q16_sparseMatrix.jpg}\\
\caption{Sparse Network: Probability of successfully detecting the partitions using the Partition-Algorithm. The x-axis corresponds to $b$ and the y-axis corresponds to $a$. The decision boundaries are overlaid in red. The community-recovery algorithm is implemented only for case where $0\leq q<p \leq 1<$. The decision boundary lying inside the Dark region has $(a,b)$ values that violate the $q<p$ requirement, and hence, can be ignored. }
\label{fig:SparseNetwork}
\end{figure}


\begin{lstlisting}
n = 300;
numTrials = 20;
alphaMax = 70;
betaMax = 50;
alphaMin = 5;
betaMin = 1;

Alpha = alphaMin:1:alphaMax;
Beta = betaMin:1:betaMax;

overlapMatrix = zeros(length(Alpha), length(Beta));

for i=1:length(Alpha)
    for j= 1:length(Beta)
        
        alpha = Alpha(i);
        beta = Beta(j);
        
% %         % Dense Matrix
% %         p = alpha/n*log(n);
% %         q = beta/n*log(n);
         
        % Sparse Matrix
        p = alpha/n;
        q = beta/n;
        
        overlapScore = zeros(1,numTrials);
        % ---------------------------------------------------------
        % We need to run the algorithm only if q<p
        % We set the default values to zero!
        if (q<p)
         for iter = 1:numTrials
             [A, w] = getPartitionGraphModel(n,p,q);
             w_pred = runPartitionAlgo(A);
             overlapScore(iter) = getPartitionOverlap(w, w_pred);
         end 
        end
        % ---------------------------------------------------------
         % store the avg. Overlap score for given (alpha,beta)
         overlapMatrix(i,j) = sum(overlapScore)/ numTrials;
    end 
end
% % % 
% % % % ----------------------------------------------------
% % % % Plot the following curve on the 
% % % % alpha - beta > sqrt(0.5*(alpha + beta))*2 / sqrt(log(n))
% % % % k = 2/sqrt(log n)
% % % % alpha > 0.5*( (2*beta + k^2/2)\pm k/2 sqrt(16*beta + k^2/2)   )
% % % 
% % % k = 2/sqrt(log(n));
% % % Alpha1 = 0.5*((2*Beta + k^2/2) + (k/2)*sqrt(16*Beta + k^2));
% % % Alpha2 = 0.5*((2*Beta + k^2/2) - (k/2)*sqrt(16*Beta + k^2));
% % % % ----------------------------------------------------

% ----------------------------------------------------
% How does this change for the sparse matrix?
% SPARSE MATRIX BOUNDARIES
Alpha1 = (Beta + 1) + sqrt(1+ 4*Beta);
Alpha2 = (Beta + 1) - sqrt(1+ 4*Beta);

% ----------------------------------------------------
close all
fig1 = figure(1)
% Plot the image as a GRAYSCALE
betaDim = [betaMin betaMax];
alphaDim = [alphaMin alphaMax];
img = imagesc(betaDim, alphaDim, overlapMatrix);
colormap(gray);
colorbar;
hold on
ax2 = plot(Beta,Alpha1,'r');
% colormap(ax2, parula )
hold on
ax3 = plot(Beta,Alpha2,'r');
% colormap(ax3, parula )
hold off
title('Sparse Matrix: Probability of successfully detecting Partitions')
% title('Dense Matrix: Probability of successfully detecting Partitions')
\end{lstlisting}

\hrulefill



\textbf{Q 18. Zachary's Karate Club}

Overlap Score $= 1$. The code used for implementing the same is described below.
\begin{lstlisting}
load zachary.mat

% From the question/visual graph: Identify true_Partition
nodeIDs = 1:34;
idx_teamA = [25 26 28 32 24 29 30 27 10 34 9 21 33 31 19 23 15 16]
idx_teamB = setdiff(1:34,idx_teamA)

truePartition = ones(size(nodeIDs));
truePartition(idx_teamB) = -1*ones(size(idx_teamB));


% Implement the algorithm to detect the partitions.

M = (A ~= 0 );
adjMatrix = zeros(size(A));
adjMatrix(M) = 1;

estPartition = runPartitionAlgo(adjMatrix);
cluster1 = (estPartition <0);
cluster1_idx_est = cluster1.*nodeIDs;

cluster2 = (estPartition >0);
cluster2_idx_est = cluster2.*nodeIDs;

overlapScore = getPartitionOverlap(truePartition, estPartition)

\end{lstlisting}

\hrulefill

\end{document}